\documentclass{article}

\usepackage{geometry}
\usepackage{amsmath}
\usepackage{graphicx}
\usepackage{listings}
\usepackage{hyperref}
\usepackage{multicol}
\usepackage{fancyhdr}
\pagestyle{fancy}
\hypersetup{ colorlinks=true, linkcolor=black, filecolor=magenta, urlcolor=cyan}
\geometry{ a4paper, total={170mm,257mm}, top=20mm, right=20mm, bottom=20mm, left=20mm}
\setlength{\parindent}{0pt}
\setlength{\parskip}{1em}
\renewcommand{\headrulewidth}{0pt}
\lhead{Competitive Programming - Arkavidia V}
\fancyfoot[CE,CO]{\thepage}
\lstset{
    basicstyle=\ttfamily\small,
    columns=fixed,
    extendedchars=true,
    breaklines=true,
    tabsize=2,
    prebreak=\raisebox{0ex}[0ex][0ex]{\ensuremath{\hookleftarrow}},
    frame=none,
    showtabs=false,
    showspaces=false,
    showstringspaces=false,
    prebreak={},
    keywordstyle=\color[rgb]{0.627,0.126,0.941},
    commentstyle=\color[rgb]{0.133,0.545,0.133},
    stringstyle=\color[rgb]{01,0,0},
    captionpos=t,
    escapeinside={(\%}{\%)}
}

\begin{document}

\begin{center}
    \section*{D. Danau Ganesha}

    \begin{tabular}{ | c c | }
        \hline
        Batas Waktu  & 1s \\
        Batas Memori & 512MB \\
        \hline
    \end{tabular}
\end{center}

\subsection*{Deskripsi}

Deskripsi soal ini sama dengan soal E, namun berbeda di kalimat terakhir.

Di negeri Ganesha, terdapat sebuah danau.
Danau ini memiliki $N$ batu yang bersebelahan membentuk satu garis lurus, dinomori dari $1$ hingga $N$.

Kini seekor katak berada di daratan.
Katak ini spesies unik, karena ia hanya akan lompat ke batu dengan jarak ganjil.
Artinya, jika ia sedang di daratan, ia hanya lompat ke batu $1, 3, 5$, dan seterusnya.
Jika ia sedang berada di batu ke $x$, ia hanya lompat ke batu $x+1, x+3, x+5$, dan seterusnya.
Katak hanya akan melompat menjauhi daratan, sehingga jika ia lompat dari batu $X$ ke $Y$, pastilah $X < Y$.

Kini, Anda diminta menghitung berapa banyak rute katak yang mungkin.
Dua rute dianggap berbeda jika satu batu diinjak di salah satu rute, namun tidak di rute yang lain.
Katak boleh berhenti di mana saja dari batu ke-$1$ hingga $N$, \textbf{namun ia hanya boleh melompat tepat $K$ kali}.

\subsection*{Format Masukan}

Baris pertama tediri dari sebuah bilangan $T$ ($1 \leq T \leq 100$), menyatakan banyak kasus uji.
Tiap kasus uji terdiri dari satu baris berisi bilangan bulat $N$ dan $K$ ($1 \leq K \leq N \leq 100.000$) menyatakn banyaknya batu dan maksimum banyaknya lompatan.

\subsection*{Format Keluaran}

Untuk setiap kasus uji, tuliskan dalam satu baris, banyaknya rute yang mungkin. Tuliskan nilai jawaban dimodulo $1.000.000.007$.

\begin{multicols}{2}
\subsection*{Contoh Masukan}
\begin{lstlisting}
2
1 1
4 2
5 3
\end{lstlisting}
\columnbreak
\subsection*{Contoh Keluaran}
\begin{lstlisting}
1
3
4
\end{lstlisting}
\vfill
\null
\end{multicols}

\subsection*{Penjelasan}

Pada kasus uji pertama, hanya ada rute yang mungkin, yakni \lstinline{[1]}.

Pada kasus uji kedua, contoh rute katak antara lain \lstinline{[1 2]}, \lstinline{[1 4]}, dan \lstinline{[3 4]}.

Pada kasus uji ketiga, contoh rute katak antara lain \lstinline{[1 2 3]}, \lstinline{[1 2 5]}, \lstinline{[1 4 5]}, dan \lstinline{[3 4 5]}.

\pagebreak

\end{document}