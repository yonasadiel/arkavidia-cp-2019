\documentclass{article}

\usepackage{geometry}
\usepackage{amsmath}
\usepackage{graphicx}
\usepackage{listings}
\usepackage{hyperref}
\usepackage{multicol}
\usepackage{fancyhdr}
\pagestyle{fancy}
\hypersetup{ colorlinks=true, linkcolor=black, filecolor=magenta, urlcolor=cyan}
\geometry{ a4paper, total={170mm,257mm}, top=20mm, right=20mm, bottom=20mm, left=20mm}
\setlength{\parindent}{0pt}
\setlength{\parskip}{1em}
\renewcommand{\headrulewidth}{0pt}
\lhead{Competitive Programming - Arkavidia V}
\fancyfoot[CE,CO]{\thepage}

\begin{document}

\begin{center}
    \section*{H. Hydra} % ganti judul soal

    \begin{tabular}{ | c c | }
        \hline
        Batas Waktu  & 4s \\    % jangan lupa ganti time limit
        Batas Memori & 256MB \\  % jangan lupa ganti memory limit
        \hline
    \end{tabular}
\end{center}

\subsection*{Deskripsi}

Hydra menyerang kota! Arvy sebagai ksatria yang paling kuat di negeri ITB, dipercaya untuk menumpas Hydra.
Besar kekuatan Hydra berbanding dengan banyak kepala yg ia miliki.
Saat ini, Hydra memiliki 1 kepala. Namun Arvy mendengar, kepala Hydra dapat berlipat ganda.
Tiap malam, 1 kepala akan bercabang menjadi $X$ kepala.
Sebagai contoh, bila $X = 3$, Hydra akan memiliki 3 kepala di hari kedua, 9 kepala di hari ketiga, dan seterusnya.

Tapi Arvy memiliki rencana lain. Arvy mau menjinakkan Hydra supaya bisa digunakan untuk menyerang kerajaan lain.
Namun setelah dijinakkan, kepala Hydra tidak akan berlipat ganda.

Untuk menjinakkan Hydra, Arvy harus memberi Hydra buah kesukaannya, apel.
Tentu, semua kepala Hydra harus mendapat apel sama banyak.
Arvy memiliki banyak sekali apel.
Tepatnya, banyaknya apel yang Arvy miliki adalah $A_1! \times A_2! \times A_3! \times \dots \times A_N!$.
Semua apel ini harus digunakan tanpa sisa, atau raja bisa marah pada Arvy.

Jadi, pada hari keberapa Arvy harus menjinakkan Hydra supaya mendapat kondisi Hydra dengan jumlah kepala sebanyak mungkin?

\subsection*{Format Masukan}
Baris pertama terdiri dari satu bilangan bulat positif $T$ ($1 \leq T \leq 10$), menyatakan banyaknya kasus uji.
Baris kedua terdiri dari dua bilangan $X$ ($1 \leq X \leq {10}^{10}$) dan $N$ ($1 \leq N \leq 100.000$), banyaknya cabang kepala hydra dan menyatakan banyaknya bilangan $A$, sesuai deskripsi di atas.
Baris berikutnya terdiri dari $N$ bilangan $A_1, A_2, \dots, A_N$ ($1 \leq A_i \leq {10}^{13}$).

\subsection*{Format Keluaran}
Untuk tiap kasus uji, keluarkan satu bilangan, menyatakan pada hari ke berapa Arvy harus mejinakkan Hydra supaya mendapat jumlah kepala sebanyak mungkin.

\begin{multicols}{2}
\subsection*{Contoh Masukan}
\begin{lstlisting}
2
2 3
2 2 2
2 4
1 2 3 4
\end{lstlisting}
\columnbreak
\subsection*{Contoh Keluaran}
\begin{lstlisting}
3
5
\end{lstlisting}
\vfill
\null
\end{multicols}

\subsection*{Penjelasan}
Pada kasus uji pertama, banyaknya apel ada $2! \times 2! \times 2! = 8$.
Pada hari ketiga, Hydra memiliki $8$ kepala dan apel masih bisa dibagi sama rata dengan masing-masing kepala mendapat $1$ apel.
Pada hari keempat, Hydra memiliki $16$ kepala dan apel sudah tidak bisa dibagi sama rata tanpa sisa.
Jadi, Arvy harus memberikan apel pada hari ketiga.

Pada kasus uji kedua, banyaknya apel ada $1! \times 2! \times 3! \times 4! = 288$.
Pada hari kelima, Hydra memiliki $32$ kepala dan apel masih bisa dibagi sama rata dengan masing-masing kepala mendapat $9$ apel.
Pada hari keenam, Hydra memiliki $64$ kepala dan apel sudah tidak bisa dibagi sama rata tanpa sisa.
Jadi, Arvy harus memberikan apel pada hari kelima.

\pagebreak

\end{document}