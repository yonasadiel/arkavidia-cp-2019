\documentclass{article}

\usepackage{geometry}
\usepackage{amsmath}
\usepackage{graphicx}
\usepackage{listings}
\usepackage{hyperref}
\usepackage{multicol}
\usepackage{fancyhdr}
\pagestyle{fancy}
\hypersetup{ colorlinks=true, linkcolor=black, filecolor=magenta, urlcolor=cyan}
\geometry{ a4paper, total={170mm,257mm}, top=20mm, right=20mm, bottom=20mm, left=20mm}
\setlength{\parindent}{0pt}
\setlength{\parskip}{1em}
\renewcommand{\headrulewidth}{0pt}
\lhead{Competitive Programming - Arkavidia V}
\fancyfoot[CE,CO]{\thepage}

\begin{document}

\begin{center}
    \section*{E. Elegant Sequence} % ganti judul soal

    \begin{tabular}{ | c c | }
        \hline
        Batas Waktu  & 1s \\    % jangan lupa ganti time limit
        Batas Memori & 64MB \\  % jangan lupa ganti memory limit
        \hline
    \end{tabular}
\end{center}

\subsection*{Deskripsi}


Diberikan sebuah array terurut berukuran $N$ dengan elemen bernilai $K_1,K_2,\dots,K_N$
\textit{Elegant sequence} adalah array yang elemen terkecilnya adalah bilangan ganjil dan selisih dua elemen yang bersebelahan juga berupa bilangan ganjil.
Untuk setiap array terdapat bilangan S yang menyatakan banyaknya subbagian dari array yang merupakan deret elegan.

\subsection*{Format Masukan}

Baris pertama tediri dari sebuah bilangan $T$ ($1 \leq T \leq 100$), menyatakan banyak kasus uji.
Tiap kasus uji terdiri dari bilangan $N$ ($1 \leq N \leq 1000$) .
Baris berikutnya adalah bilangan $K_1,K_2,\dots,K_N$ ($1 \leq K_1,K_2,\dots,K_N \leq 1000$).

\subsection*{Format Keluaran}

Untuk setiap kasus uji, tuliskan bilangan $S$ yang merupakan banyak \textit{Elegant sequence} dari array tersebut.



\begin{multicols}{2}
\subsection*{Contoh Masukan}
\begin{lstlisting}
3
4 
1 3 5 7
5
4 5 6 10 11
6
3 5 12 21 24 27
\end{lstlisting}
\columnbreak
\subsection*{Contoh Keluaran}
\begin{lstlisting}
TBD
\end{lstlisting}
\vfill
\null
\end{multicols}


% \subsection*{Penjelasan}


\pagebreak

\end{document}