\documentclass{article}

\usepackage{geometry}
\usepackage{amsmath}
\usepackage{graphicx}
\usepackage{listings}
\usepackage{hyperref}
\usepackage{multicol}
\usepackage{fancyhdr}
\pagestyle{fancy}
\hypersetup{ colorlinks=true, linkcolor=black, filecolor=magenta, urlcolor=cyan}
\geometry{ a4paper, total={170mm,257mm}, top=20mm, right=20mm, bottom=20mm, left=20mm}
\setlength{\parindent}{0pt}
\setlength{\parskip}{1em}
\renewcommand{\headrulewidth}{0pt}
\lhead{Competitive Programming - Arkavidia V}
\fancyfoot[CE,CO]{\thepage}

\begin{document}

\begin{center}
    \section*{B. Bitwise}

    \begin{tabular}{ | c c | }
        \hline
        Batas Waktu  & 1s \\
        Batas Memori & 64MB \\
        \hline
    \end{tabular}
\end{center}

\subsection*{Deskripsi}

Pada \textit{programming}, terdapat operator bitwise berupa AND, OR, XOR.
Untuk lebih jelasnya, perhatikan tabel di bawah: (C menunjukkan hasil dari aplikasi operator terhadap A dan B)

\begin{multicols}{3}
    Tabel AND

    \begin{tabular} { | c c | c | }
        \hline
        A & B & C \\
        \hline
        1 & 1 & 1 \\
        1 & 0 & 0 \\
        0 & 1 & 0 \\
        0 & 0 & 0 \\
        \hline
    \end{tabular}

    Tabel OR

    \begin{tabular} { | c c | c | }
        \hline
        A & B & C \\
        \hline
        1 & 1 & 1 \\
        1 & 0 & 1 \\
        0 & 1 & 1 \\
        0 & 0 & 0 \\
        \hline
    \end{tabular}

    Tabel XOR

    \begin{tabular} { | c c | c | }
        \hline
        A & B & C \\
        \hline
        1 & 1 & 0 \\
        1 & 0 & 1 \\
        0 & 1 & 1 \\
        0 & 0 & 0 \\
        \hline
    \end{tabular}
\end{multicols}

Lalu, jika kita memiliki dua buah bilangan, kita dapat melakukan operasi bitwise dengan:
\begin{enumerate}
    \item Tulis kedua bilangan menjadi representasi binsernya.
    \item Lakukan operasi bitwise untuk tiap digit bilangan.
    \item Tulis bilangan baru hasil operasi sebagai bilangan baru.
\end{enumerate}

Sebagai contoh, \lstinline{5 AND 1 = 4}, \lstinline{3 OR 6 = 7}, dan \lstinline{3 XOR 6 = 5}.
Selain itu, operator bitwise juga dapat dilakukan secara asosiatif, artinya \lstinline{(A op B) op C} dengan \lstinline{A op (B op C)} akan menghasilkan nilai yang sama.

Kini Arvy punya tantangan untuk kalian.
Arvy memiliki bilangan $X$, $N$, dan operasi bitwise $op$.
Tugas kalian adalah menuliskan $N$ buah bilangan $A_1, A_2, \dots, A_N$ sehingga $A_1 \  op \  A_2 \  op \  \dots \  op \  A_N = X$.

\subsection*{Format Masukan}

Baris pertama terdiri dari satu bilangan bulat positif $T$ ($1 \leq T \leq 1.000$), menyatakan banyaknya kasus uji.
Tiap kasus uji terdiri dari sebuah baris berisikan bilangan $X$ ($1 \leq X \leq 1.000.000.000$), $N$ ($1 \leq N \lq 1.000$), dan perintah $op$ ("\lstinline{AND}", "\lstinline{OR}", atau "\lstinline{XOR}"), dipisahkan oleh spasi.

\subsection*{Format Keluaran}

Untuk tiap kasus uji, tuliskan dalam satu baris bilangan $A_1, A_2, \dots, A_N$ dipisahkan oleh spasi.
Setiap bilangan $A_i$ harus lebih dari $0$ dan kurang dari sama dengan $1.000.000.000$.
Jika ada lebih dari satu kemungkinan jawaban, tuliskan yang mana saja.
\\

\begin{multicols}{2}
\subsection*{Contoh Masukan}
\begin{lstlisting}
3
4 2 AND
11 3 OR
2 3 XOR
\end{lstlisting}
\columnbreak
\subsection*{Contoh Keluaran}
\begin{lstlisting}
5 4
2 3 9
6 7 3
\end{lstlisting}
\vfill
\null
\end{multicols}

\pagebreak

\end{document}