\documentclass{article}

\usepackage{geometry}
\usepackage{amsmath}
\usepackage{graphicx}
\usepackage{listings}
\usepackage{hyperref}
\usepackage{multicol}
\usepackage{fancyhdr}
\pagestyle{fancy}
\hypersetup{ colorlinks=true, linkcolor=black, filecolor=magenta, urlcolor=cyan}
\geometry{ a4paper, total={170mm,257mm}, top=20mm, right=20mm, bottom=20mm, left=20mm}
\setlength{\parindent}{0pt}
\setlength{\parskip}{1em}
\renewcommand{\headrulewidth}{0pt}
\lhead{Competitive Programming - Arkavidia V}
\fancyfoot[CE,CO]{\thepage}

\begin{document}
\begin{center}
    \section*{I. Irisan Adamantium} % ganti judul soal

    \begin{tabular}{ | c c | }
        \hline
        Batas Waktu  & 1s \\    % jangan lupa ganti time limit
        Batas Memori & 64MB \\  % jangan lupa ganti memory limit
        \hline
    \end{tabular}
\end{center}

\subsection*{Deskripsi}
Dua orang pewaris sedang memperebutkan sebuah balok Adamantium yang merupakan sebuah warisan keramat dari Kakek mereka.
Adamantium adalah sebuah logam yang sangat keras, dan biaya pemotongan nya luar biasa mahal.
Arvy diminta untuk memotong logam tersebut supaya warisan terbagi rata antara 2 orang.

Logam terdiri dari $2N$ lapisan, dan ada $N$ jenis lapisan yang tersebar acak, sehingga ada tepat 2 lapisan berbeda memiliki jenis yang sama.
Karena nilai setiap lapisan tidak diketahui pasti, satu-satunya cara memastikan agar pembagian rata adalah ketika setiap orang memiliki satu dari masing-masing jenis lapisan.
Tetapi, tingkat kekerasan Adamantium kian lapisan jauh semakin keras (tidak dipengaruhi jenis lapisan), sehingga biaya untuk memotong lapisan ke-$i$ dan lapisan $i+1$ adalah $2^i$.
Kini Arvy harus menghitung biaya pemotongan minimum untuk membagi logam tersebut kepada dua pewaris.

\subsection*{Format Masukan}
Baris pertama terdiri dari satu bilangan bulat positif $T$ ($1 \leq T \leq 100$), menyatakan banyaknya kasus uji.
Untuk tiap kasus uji, baris pertama menyatakan bilangan positif $N$ ($1 \leq N \leq 100.000$) dan baris kedua terdiri dari $N$ bilangan $A_1, A_2, \dots, A_N$ yang dipisahkan oleh spasi, menyatakan jenis tiap lapisan.

\subsection*{Format Keluaran}
Untuk tiap kasus uji, keluarkan satu baris berisi satu bilangan bulat, yakni biaya pemotongan termurah.
\\

\begin{multicols}{2}
\subsection*{Contoh Masukan}
\begin{lstlisting}
3
3
1 2 3 1 2 3
3
1 1 2 2 3 3
3
1 2 1 2 3 3
\end{lstlisting}
\columnbreak
\subsection*{Contoh Keluaran}
\begin{lstlisting}
8
42
36
\end{lstlisting}
\vfill
\null
\end{multicols}

\subsection*{Penjelasan}
Pada kasus uji pertama, dapat dilakukan 1 potongan di $i=3$, membentuk segmen \lstinline{1 2 3} dan \lstinline{1 2 3}.
Jadi, biaya pemotongan minimum adalah $2^3 = 8$.

Pada kasus uji kedua, dapat dilakukan 3 potongan di $i=1, i=4, i=5$, membentuk segmen \lstinline{1}, \lstinline{2 3}, \lstinline{1 2}, dan \lstinline{3}.
Jadi, biaya pemotongan minimum adalah $2^1 + 2^3 + 2^5 = 42$.

Pada kasus uji pertama, dapat dilakukan 2 potongan di $i=2, i=5$, membentuk segmen \lstinline{1 2}, \lstinline{3}, dan \lstinline{1 2 3}.
Jadi, biaya pemotongan minimum adalah $2^2 + 2^5 = 36$.

\pagebreak
\end{document}
