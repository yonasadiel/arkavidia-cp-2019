\documentclass{article}
\usepackage{listings}
\usepackage{geometry}
\usepackage{amsmath}
\usepackage{graphicx}
\usepackage{hyperref}
\usepackage{multicol}
\usepackage{fancyhdr}
\pagestyle{fancy}
\hypersetup{ colorlinks=true, linkcolor=black, filecolor=magenta, urlcolor=cyan}
\geometry{ a4paper, total={170mm,257mm}, top=20mm, right=20mm, bottom=20mm, left=20mm}
\setlength{\parindent}{0pt}
\setlength{\parskip}{1em}
\renewcommand{\headrulewidth}{0pt}
\lhead{Competitive Programming - Arkavidia V}
\fancyfoot[CE,CO]{\thepage}
\lstset{
    basicstyle=\ttfamily\small,
    columns=fixed,
    extendedchars=true,
    breaklines=true,
    tabsize=2,
    prebreak=\raisebox{0ex}[0ex][0ex]{\ensuremath{\hookleftarrow}},
    frame=none,
    showtabs=false,
    showspaces=false,
    showstringspaces=false,
    prebreak={},
    keywordstyle=\color[rgb]{0.627,0.126,0.941},
    commentstyle=\color[rgb]{0.133,0.545,0.133},
    stringstyle=\color[rgb]{01,0,0},
    captionpos=t,
    escapeinside={(\%}{\%)}
}

\begin{document}
\begin{center}
    \section*{B. Barisan yang Indah}

    \begin{tabular}{ | c c | }
        \hline
        Batas Waktu  & 1s \\
        Batas Memori & 512MB \\
        \hline
    \end{tabular}
\end{center}

\subsection*{Deskripsi}
Sebuah barisan dapat dikatakan indah jika barisan tersebut memiliki $M$ elemen, dan masing-masing elemennya bernilai lebih besar dari nilai ($E_1$ \& $E_2$ \& $\dots$ \& $E_M$), di mana $E_i$ menyatakan elemen ke-$i$ dari barisan tersebut dan \& adalah operator bit AND.

Diberikan sebuah barisan $A$ terdiri dari $N$ buah bilangan bulat, tentukan banyaknya $(i, j)$ sehingga $i < j$ dan barisan $A_i, A_{i+1}, \dots A_j$ merupakan barisan indah.

\subsection*{Format Masukan}
Baris pertama terdiri dari satu bilangan bulat positif $T$ ($1 \leq T \leq 10$), menyatakan banyaknya kasus uji.
Tiap kasus uji diawali dengan sebuah baris berisikan bilangan bulat $N$ ($1 \leq N \leq 100.000$).
Baris berikutnya terdiri dari $N$ bilangan, menyatakan nilai dari $A_1, A_2, \dots A_N$ ($0 \leq A_i \leq 1.000.000.000$).

\subsection*{Format Keluaran}
Untuk tiap kasus uji, tuliskan dalam sebuah baris, banyaknya pasangan $(i, j)$ yang memenuhi deskripsi di atas.
\\

\begin{multicols}{2}
\subsection*{Contoh Masukan}
\begin{lstlisting}
1
4
1 7 6 4
\end{lstlisting}
\columnbreak
\subsection*{Contoh Keluaran}
\begin{lstlisting}
2
\end{lstlisting}
\vfill
\null
\end{multicols}


\subsection*{Penjelasan}
Pada kasus uji pertama, kandidat $(i, j)$ antara lain:

\begin{enumerate}
    \setlength\itemsep{0pt}
    \item $(1, 2)$, merepresentasikan $1, 7$ dengan nilai AND $1$.
    \item $(1, 3)$, merepresentasikan $1, 7, 6$ dengan nilai AND $0$.
    \item $(1, 4)$, merepresentasikan $1, 7, 6, 4$ dengan nilai AND $0$.
    \item $(2, 3)$, merepresentasikan $7, 6$ dengan nilai AND $6$.
    \item $(2, 4)$, merepresentasikan $7, 6, 4$ dengan nilai AND $4$.
    \item $(3, 4)$, merepresentasikan $6, 4$ dengan nilai AND $4$.
\end{enumerate}

Dari semua kandidat, hanya nomor $2$ dan $3$ yang memenuhi.

\pagebreak
\end{document}
