\documentclass{article}

\usepackage{geometry}
\usepackage{amsmath}
\usepackage{graphicx}
\usepackage{listings}
\usepackage{hyperref}
\usepackage{multicol}
\usepackage{fancyhdr}
\pagestyle{fancy}
\hypersetup{ colorlinks=true, linkcolor=black, filecolor=magenta, urlcolor=cyan}
\geometry{ a4paper, total={170mm,257mm}, top=20mm, right=20mm, bottom=20mm, left=20mm}
\setlength{\parindent}{0pt}
\setlength{\parskip}{1em}
\renewcommand{\headrulewidth}{0pt}
\lhead{Competitive Programming - Arkavidia V}
\fancyfoot[CE,CO]{\thepage}

\begin{document}

\begin{center}
    \section*{F. Fantasy Land} % ganti judul soal

    \begin{tabular}{ | c c | }
        \hline
        Batas Waktu  & 1s \\    % jangan lupa ganti time limit
        Batas Memori & 64MB \\  % jangan lupa ganti memory limit
        \hline
    \end{tabular}
\end{center}

\subsection*{Deskripsi}

Ketika Arvy berulang tahun yang ke-18, ia berencana mentraktir teman-temannya untuk pergi ke wahana bermain \textit{Fantasy Land}.
Namun, wahana ini memiliki biaya tiket masuk yang unik.
Pada loket masuk terdapat $N$ pilihan paket.
Tiap paket berisikan $K$ dan $P$, yang artinya Arvy dapat membeli tiket seharga $P$ untuk $K$ orang.
Tiap paket bisa dibeli lebih dari sekali, tapi Arvy tidak ingin membeli tiket lebih dari $M$ orang.
Karena Arvy yang membayar, tentu saja ia ingin biaya yang minimum agar ia dan teman-temannya dapat masuk ke wahana bermain tersebut.


\subsection*{Format Masukan}

Baris pertama terdiri dari satu bilangan bulat positif $T$ ($1 \leq T \leq 10$), menyatakan banyaknya kasus uji.
Tiap kasus uji terdiri dari bilangan $N$ dan $Q$ ($1 \leq N, Q \leq 1.000$), dengan $N$ adalah banyak pilihan paket dan $Q$ adalah banyak pertanyaan.
$N$ baris berikutnya terdiri atas dua bilangan $K$ ($1 \leq K \leq 1.000$) dan $P$ ($1 \leq P \leq 1.000.000$), dengan $K$ menyatakan jumlah orang dan $P$ menyatakan biaya masuknya.
$Q$ baris berikutnya terdiri atas bilangan $M$ ($1 \leq M \leq 1.000$), dengan $M$ menyatakan banyaknya orang yang akan masuk (Arvy dan teman-temannya).
Dijamin ada paket di mana $K = 1$.

\subsection*{Format Keluaran}

Untuk setiap kasus uji, tuliskan $Q$ baris, masing-masing berisi satu bilangan yang menyatakan biaya minimum yang perlu dibayar untuk $M$ orang dapat masuk ke \textit{Fantasy Land}.

\begin{multicols}{2}
\subsection*{Contoh Masukan}
\begin{lstlisting}
1
7 3
1 1000
2 2500
3 3300
4 3900
5 4700
6 6400
7 7500
3
7
12
\end{lstlisting}
\columnbreak
\subsection*{Contoh Keluaran}
\begin{lstlisting}
3000
6700
11400
\end{lstlisting}
\vfill
\null
\end{multicols}


\subsection*{Penjelasan}

Saat $M=3$, daripada langsung membeli paket 3 orang seharga 3300, lebih baik membeli paket 1 orang 1000 sebanyak tiga kali.
Saat $M=7$, daripada langsung membeli paket 7 orang seharga 7500, lebih baik membeli paket 1 orang dua kali dan paket 5 orang sekali.

\pagebreak

\end{document}