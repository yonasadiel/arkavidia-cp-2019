\documentclass{article}

\usepackage{geometry}
\usepackage{amsmath}
\usepackage{graphicx}
\usepackage{listings}
\usepackage{hyperref}
\usepackage{multicol}
\usepackage{fancyhdr}
\pagestyle{fancy}
\hypersetup{ colorlinks=true, linkcolor=black, filecolor=magenta, urlcolor=cyan}
\geometry{ a4paper, total={170mm,257mm}, top=20mm, right=20mm, bottom=20mm, left=20mm}
\setlength{\parindent}{0pt}
\setlength{\parskip}{1em}
\renewcommand{\headrulewidth}{0pt}
\lhead{Competitive Programming - Arkavidia V}
\fancyfoot[CE,CO]{\thepage}

\begin{document}

\begin{center}
    \section*{G. Gaji Perusahaan}

    \begin{tabular}{ | c c | }
        \hline
        Batas Waktu  & 1s \\
        Batas Memori & 64MB \\
        \hline
    \end{tabular}
\end{center}

\subsection*{Deskripsi}

Arvy mengikuti tur ke sebuah perusahaan.
Perusahaan ini menarik, karena gaji tiap karyawan merupakan salah satu angka favorit CEO perusahaan.
Kriteria untuk gaji seorang karyawan adalah:
\begin{itemize}
    \item terdiri dari $N$ digit,
    \item tidak ada karyawan lain dengan gaji yang sama,
    \item merupakan bilangan kelipatan 3 (tiga), dan
    \item mengandung setidaknya sebuah angka 3 (tiga).
\end{itemize}
Kini Arvy penasaran, berapa maksimal banyaknya karyawan di perusahaan ini sampai tidak ada lagi angka yang memenuhi kriteria di atas?

\subsection*{Format Masukan}

Baris pertama terdiri dari satu bilangan bulat positif $T$ ($1 \leq T \leq 100.000$), menyatakan banyaknya kasus uji.
Tiap kasus uji terdiri dari satu bilangan $N$ ($1 \leq N \leq 10^{12}$), menyatakan banyaknya digit gaji perusahaan.

\subsection*{Format Keluaran}

Untuk tiap kasus uji, tuliskan sebuah baris, menyatakan banyaknya karyawan maksimal yang bisa dimiliki perusahaan tersebut tanpa ada 2 karyawan dengan gaji yang sama, dimodulo ${10}^9+7$.
\\

\begin{multicols}{2}
\subsection*{Contoh Masukan}
\begin{lstlisting}
3
2
5
7
\end{lstlisting}
\columnbreak
\subsection*{Contoh Keluaran}
\begin{lstlisting}
6
12504
1582824
\end{lstlisting}
\vfill
\null
\end{multicols}

\subsection*{Penjelasan}
Pada kasus uji pertama, daftar bilangan yang memenuhi antara lain: $30, 33, 36, 39, 63, 93$

\pagebreak

\end{document}