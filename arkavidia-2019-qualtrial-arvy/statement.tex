\documentclass{article}

\usepackage{geometry}
\usepackage{amsmath}
\usepackage{graphicx}
\usepackage{listings}
\usepackage{hyperref}
\usepackage{multicol}
\usepackage{fancyhdr}
\pagestyle{fancy}
\hypersetup{ colorlinks=true, linkcolor=black, filecolor=magenta, urlcolor=cyan}
\geometry{ a4paper, total={170mm,257mm}, top=20mm, right=20mm, bottom=20mm, left=20mm}
\setlength{\parindent}{0pt}
\setlength{\parskip}{1em}
\renewcommand{\headrulewidth}{0pt}
\lhead{Competitive Programming - Arkavidia V}
\fancyfoot[CE,CO]{\thepage}

\begin{document}

\begin{center}
    \section*{A. Arvy}

    \begin{tabular}{ | c c | }
        \hline
        Batas Waktu  & 1s \\
        Batas Memori & 64MB \\
        \hline
    \end{tabular}
\end{center}

\subsection*{Deskripsi}

\begin{center}
\includegraphics[width=200px]{arvy}
\end{center}

Halo, aku Arvy!
Arvy adalah maskot Arkavidia tahun ini.
Dalam bahasa Jerman, arti dari Arvy adalah "\textit{friend of the people}".
Seperti namanya, Arvy akan menemani kalian sampai akhir Arkavidia 5.0.

Kalau kamu mau berteman dengan Arvy, sampaikan salammu padanya!

\subsection*{Format Masukan}

Tidak ada masukan.

\subsection*{Format Keluaran}

Keluarkan sebuah baris bertuliskan "\lstinline{Halo, Arvy!}" (tanpa tanda kutip).
\\

\begin{multicols}{2}
\subsection*{Contoh Masukan}
\begin{lstlisting}

\end{lstlisting}
\columnbreak
\subsection*{Contoh Keluaran}
\begin{lstlisting}
Halo, Arvy!
\end{lstlisting}
\vfill
\null
\end{multicols}

% \subsection*{Penjelasan}
% Jika dibutuhkan, tambahkan penjelasan di sini

\pagebreak

\end{document}