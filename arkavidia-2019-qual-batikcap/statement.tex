\documentclass{article}

\usepackage{geometry}
\usepackage{amsmath}
\usepackage{graphicx}
\usepackage{listings}
\usepackage{hyperref}
\usepackage{multicol}
\usepackage{fancyhdr}
\pagestyle{fancy}
\hypersetup{ colorlinks=true, linkcolor=black, filecolor=magenta, urlcolor=cyan}
\geometry{ a4paper, total={170mm,257mm}, top=20mm, right=20mm, bottom=20mm, left=20mm}
\setlength{\parindent}{0pt}
\setlength{\parskip}{1em}
\renewcommand{\headrulewidth}{0pt}
\lhead{Competitive Programming - Arkavidia V}
\fancyfoot[CE,CO]{\thepage}

\begin{document}
\begin{center}
    \section*{B. Batik Cap}
    \begin{tabular}{ | c c | }
        \hline
        Batas Waktu  & 1s \\
        Batas Memori & 64MB \\
        \hline
    \end{tabular}
\end{center}

\subsection*{Deskripsi}
Arvy, seorang pembatik yang hemat, ingin membuat cap untuk dua pola batik yang akan ia buat.
Pola batik Arvy dapat digambarkan sebagai sebuah \textit{string} yang terdiri dari $N$ karakter.
Sebuah pola batik $B$ dapat dibentuk oleh pola cap $C$ jika $B$ dapat dibentuk dengan mengulang pola $C$.
Sebagai contoh, \lstinline{ababab} dapat dibentuk dari pola \lstinline{ab} atau \lstinline{ababab}, namun tidak \lstinline{aba} atau \lstinline{a}.

Ia hanya ingin membuat satu cap, namun tetap dapat membentuk kedua pola batik.
Ia juga ingin jumlah penggunaan cap untuk membuat kedua pola batik harus sesedikit mungkin.
Tentukan pola cap seperti apa yang harus ia buat!

\subsection*{Format Masukan}
Baris pertama terdiri dari satu bilangan bulat positif $T$ ($1 \leq T \leq 100$), menyatakan banyaknya kasus uji.
Tiap kasus uji terdiri dari 2 string yang menunjukkan pola batik yang ingin dibuat, dipisahkan oleh baris baru.
Pola batik dijamin hanya terdiri dari huruf non-kapital alfabet Bahasa Indonesia (a-z).
Dijamin jumlah panjang pola batik untuk semua kasus uji tidak lebih dari $100.000$ karakter.

\subsection*{Format Keluaran}
Untuk tiap kasus uji, tuliskan dalam satu baris, satu pola cap batik yang dapat membentuk kedua pola batik dengan jumlah pengecapan paling minimum.
Jika tidak ada pola yang memenuhi, tuliskan \lstinline{-} (strip).
\\

\begin{multicols}{2}
\subsection*{Contoh Masukan}
\begin{lstlisting}
3
ababab
abab
bbcdbbcd
bbcd
aaa
bb
\end{lstlisting}
\columnbreak
\subsection*{Contoh Keluaran}
\begin{lstlisting}
ab
bbcd
-
\end{lstlisting}
\vfill
\null
\end{multicols}

\pagebreak
\end{document}