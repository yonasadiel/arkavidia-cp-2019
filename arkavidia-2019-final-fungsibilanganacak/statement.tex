\documentclass{article}

\usepackage{geometry}
\usepackage{amsmath}
\usepackage{graphicx}
\usepackage{listings}
\usepackage{hyperref}
\usepackage{multicol}
\usepackage{fancyhdr}
\pagestyle{fancy}
\hypersetup{ colorlinks=true, linkcolor=black, filecolor=magenta, urlcolor=cyan}
\geometry{ a4paper, total={170mm,257mm}, top=20mm, right=20mm, bottom=20mm, left=20mm}
\setlength{\parindent}{0pt}
\setlength{\parskip}{1em}
\renewcommand{\headrulewidth}{0pt}
\lhead{Competitive Programming - Arkavidia V}
\fancyfoot[CE,CO]{\thepage}
\lstset{
    basicstyle=\ttfamily\small,
    columns=fixed,
    extendedchars=true,
    breaklines=true,
    tabsize=2,
    prebreak=\raisebox{0ex}[0ex][0ex]{\ensuremath{\hookleftarrow}},
    frame=none,
    showtabs=false,
    showspaces=false,
    showstringspaces=false,
    prebreak={},
    keywordstyle=\color[rgb]{0.627,0.126,0.941},
    commentstyle=\color[rgb]{0.133,0.545,0.133},
    stringstyle=\color[rgb]{01,0,0},
    captionpos=t,
    escapeinside={(\%}{\%)}
}

\begin{document}

\begin{center}
    \section*{F. Fungsi Bilangan Acak}

    \begin{tabular}{ | c c | }
        \hline
        Batas Waktu  & 1s \\
        Batas Memori & 512MB \\
        \hline
    \end{tabular}
\end{center}

\subsection*{Deskripsi}

Arvy mengembangkan sebuah fungsi untuk membangkitkan bilangan acak.
Untuk fungsi ini, dibutuhkan parameter atau \textit{seed} $A, B, N, l, r$.
Pertama, ia akan menulis nilai $A / B$ hingga $N$ bilangan di belakang koma.
Lalu, bilangan acak diambil dari digit ke-$l$ hingga $r$ di belakang koma.
Tentukanlah bilangan acak yang akan dibangkitkan!

\subsection*{Format Masukan}
Baris pertama terdiri dari satu bilangan bulat positif $T$ ($1 \leq T \leq 10$), menyatakan banyaknya kasus uji.
Tiap kasus uji dimulai dengan sebuah baris berisikan nilai $N$ ($1 \leq N \leq 10^{9}$), $A$, dan $B$ ($1 \leq A, B \leq 10^{18}$) sesuai deskripsi di atas.
Baris berikutnya terdiri dari sebuah bilangan $Q$ ($1 \leq Q \leq 100.000$), menyatakan banyaknya bilangan acak.
$Q$ baris berikutnya terdiri dari dua bilangan, yakni $l$ dan $r$ ($1 \leq l \leq r \leq N$), sesuai deskripsi di atas.

\subsection*{Format Keluaran}
Untuk tiap kasus uji, keluarkan $Q$ baris berisi bilangan acak yang diminta. Karena bilangan bisa sangat besar, tuliskan nilainya dimodulo $1.000.000.007$

\begin{multicols}{2}
\subsection*{Contoh Masukan}
\begin{lstlisting}
2
5 1 2
3
1 2
1 3
2 5
20 1000000008 9999999999
3
1 4
5 10
1 10
\end{lstlisting}
\columnbreak
\subsection*{Contoh Keluaran}
\begin{lstlisting}
50
500
0
1000
8
1
\end{lstlisting}
\vfill
\null
\end{multicols}

\subsection*{Penjelasan}
Pada kasus uji pertama, bilangan $5$ digit yang terbentuk adalah \lstinline{50000}.

\begin{itemize}
    \setlength\itemsep{0pt}
    \item Digit pertama hingga kedua adalah \lstinline{50}, sehingga nilai bilangan acaknya $50$.
    \item Digit pertama hingga ketiga adalah \lstinline{500}, sehingga nilai bilangan acaknya $500$.
    \item Digit kedua hingga kelima adalah \lstinline{0000}, sehingga nilai bilangan acaknya $0$.
\end{itemize}

Pada kasus uji kedua, bilangan $20$ digit yang terbentuk adalah \lstinline{10000000081000000008}.

\begin{itemize}
    \setlength\itemsep{0pt}
    \item Digit pertama hingga keempat adalah \lstinline{1000}, sehingga nilai bilangan acaknya $1000$.
    \item Digit kelima hingga kesepuluh adalah \lstinline{000008}, sehingga nilai bilangan acaknya $8$.
    \item Digit pertama hingga kesepuluh adalah \lstinline{1000000008}, sehingga nilai bilangan acaknya $1$ (setelah dimodulo).
\end{itemize}

\end{document}