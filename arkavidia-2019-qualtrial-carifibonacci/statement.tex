\documentclass{article}

\usepackage{geometry}
\usepackage{amsmath}
\usepackage{graphicx}
\usepackage{listings}
\usepackage{hyperref}
\usepackage{multicol}
\usepackage{fancyhdr}
\pagestyle{fancy}
\hypersetup{ colorlinks=true, linkcolor=black, filecolor=magenta, urlcolor=cyan}
\geometry{ a4paper, total={170mm,257mm}, top=20mm, right=20mm, bottom=20mm, left=20mm}
\setlength{\parindent}{0pt}
\setlength{\parskip}{1em}
\renewcommand{\headrulewidth}{0pt}
\lhead{Competitive Programming - Arkavidia V}
\fancyfoot[CE,CO]{\thepage}

\begin{document}

\begin{center}
    \section*{C. Cari Fibonacci}

    \begin{tabular}{ | c c | }
        \hline
        Batas Waktu  & 1s \\
        Batas Memori & 64MB \\
        \hline
    \end{tabular}
\end{center}

\subsection*{Deskripsi}

Barisan fibonacci merupakan suatu barisan bilangan bulat dimana setiap suku fibonacci merupakan penjumlahan dua suku sebelumnya. Secara matematis, bilangan fibonacci didefinisikan sebagai berikut:

\begin{align*}
F_1 & = 1 \\
F_2 & = 1 \\
F_n & = F_{n-1} + F_{n-2}
\end{align*}

Diberikan sebuah bilangan bulat $N$, tentukan nilai $F_N$ modulo $1.000.000.007$.

\subsection*{Format Masukan}

Baris pertama terdiri dari satu bilangan bulat positif $T$ ($1 \leq T \leq 100.000$), menyatakan banyaknya kasus uji.
Tiap kasus uji terdiri dari satu baris berisikan bilangan $N$ ($1 \leq N \leq 100.000$).

\subsection*{Format Keluaran}

Untuk tiap kasus uji, tuliskan sebuah baris berisi nilai dari $F_N$, dimodulo 1.000.000.007.
\\

\begin{multicols}{2}
\subsection*{Contoh Masukan}
\begin{lstlisting}
2
1
5
\end{lstlisting}
\columnbreak
\subsection*{Contoh Keluaran}
\begin{lstlisting}
1
5
\end{lstlisting}
\vfill
\null
\end{multicols}

\pagebreak

\end{document}